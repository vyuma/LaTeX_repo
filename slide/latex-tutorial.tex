\documentclass{beamer}
\usetheme{CambridgeUS}       % テーマの選択
\usepackage{graphicx}        % 各種画像の張り込み
\usepackage{amsmath,amssymb} % 標準数式表現を拡大する
\usepackage{luatexja}        % 日本語対応
\usepackage[ipaex]{luatexja-preset} % IPAexフォントしたい

\setbeamertemplate{navigation symbols}{}      % ナビゲーションバー非表示

% \setbeamertemplate{theorems}[numbered]
% \addtobeamertemplate{theorem begin}{\normalfont}{}
% \uselanguage{japanese}
% \languagepath{japanese}
% \deftranslation[to=japanese]{Theorem}{定理}
% \deftranslation[to=japanese]{Corollary}{系}
% \deftranslation[to=japanese]{Lemma}{補題}
% \newtheorem{proposition}[theorem]{命題}
% \deftranslation[to=japanese]{Proposition}{命題}
% \deftranslation[to=japanese]{Example}{例}
% \deftranslation[to=japanese]{Examples}{例}
% \deftranslation[to=japanese]{Definition}{定義}
% \deftranslation[to=japanese]{Definitions}{定義}
% \deftranslation[to=japanese]{Problem}{問題}
% \deftranslation[to=japanese]{Solution}{解}
% \deftranslation[to=japanese]{Fact}{事実}
% \deftranslation[to=japanese]{Proof}{証明}
% \def\proofname{証明}

\title{VS Codeで\LaTeX を書く}
\subtitle{VS Codeの機能を使いこなす}
\author{リュカ, 裕磨}
\institute{ゼロイチゼミ, 学術サーバー}
\date{\today}

\begin{document}

\begin{frame}
  \titlepage
\end{frame}

\section{\LaTeX とは}
\subsection{}

\begin{frame}
  \frametitle{\LaTeX のインストール}
  \LaTeX のインストールは非常に時間がかかるため始めにやっておく
  %リンクとQRコード
\end{frame}

\begin{frame}
  \frametitle{なぜ\LaTeX を使うのか}
  %wordとの比較を交えつつ論文執筆のデファクトスタンダードとなっていること、数式の組版処理が優れていること、修正、再利用が容易なことなどを話す
\end{frame}

\begin{frame}
  \frametitle{\LaTeX の記法}


\end{frame}

\begin{frame}
  \frametitle{latexmkの設定}


\end{frame}

\begin{frame}
  \frametitle{LaTeX workshopの設定}


\end{frame}

\begin{frame}
  \frametitle{Ultra Math Previewのインストール}
  

\end{frame}

\end{document}