\documentclass{beamer}
\usetheme{CambridgeUS}       % テーマの選択
\usepackage{graphicx}        % 各種画像の張り込み
\usepackage{amsmath,amssymb} % 標準数式表現を拡大する
\usepackage{luatexja}        % 日本語対応
\usepackage[ipaex]{luatexja-preset} % IPAexフォントしたい

\setbeamertemplate{navigation symbols}{}      % ナビゲーションバー非表示
\usepackage{listings,jvlisting} %日本語のコメントアウトをする場合jvlisting(もしくはjlisting)が必要
%ここからソースコードの表示に関する設定
% \usepackage{lua-visual-debug}
\lstset{
	stringstyle={\ttfamily},
	commentstyle={\ttfamily},
	basicstyle={\ttfamily},
	columns=fixed,
  frame={tb},
  breaklines=true,
  columns=[l]{fullflexible},
	numbers=left,%行数を表示したければonにする
	numberstyle={\scriptsize},
  xrightmargin=0em,
  xleftmargin=3em,
  stepnumber=1,
  numbersep=1em,
	tabsize=2,
  lineskip=-0.5ex,
  backgroundcolor=\color{white}
}

\usepackage{bxtexlogo} % platex, lualatexのロゴを使うために必要

% \setbeamertemplate{theorems}[numbered]
% \addtobeamertemplate{theorem begin}{\normalfont}{}
% \uselanguage{japanese}
% \languagepath{japanese}
% \deftranslation[to=japanese]{Theorem}{定理}
% \deftranslation[to=japanese]{Corollary}{系}
% \deftranslation[to=japanese]{Lemma}{補題}
% \newtheorem{proposition}[theorem]{命題}
% \deftranslation[to=japanese]{Proposition}{命題}
% \deftranslation[to=japanese]{Example}{例}
% \deftranslation[to=japanese]{Examples}{例}
% \deftranslation[to=japanese]{Definition}{定義}
% \deftranslation[to=japanese]{Definitions}{定義}
% \deftranslation[to=japanese]{Problem}{問題}
% \deftranslation[to=japanese]{Solution}{解}
% \deftranslation[to=japanese]{Fact}{事実}
% \deftranslation[to=japanese]{Proof}{証明}
% \def\proofname{証明}

\title{VS Codeで\LaTeX を書く}
\subtitle{VS Codeの機能を使いこなす}
\author{リュカ, 裕磨}
\institute{ゼロイチゼミ, 学術サーバー}
\date{\today}

\begin{document}

\begin{frame}
  \titlepage
\end{frame}

\section{\LaTeX とは}
\subsection{}

\begin{frame}
  \frametitle{TeXliveの導入}
  \begin{enumerate}
	\item TeXliveのインストーラーを次のページからダウンロードする。
  \begin{quote}
    \centering
    \url{https://www.tug.org/texlive/acquire-netinstall.html}
  \end{quote}
  をクリックする。
  \item ページ上のリンク install-tl-windows.exeをクリックする。
  \item ダウンロードを終えたら実行する。
\end{enumerate}
\end{frame}
\begin{frame}
    \frametitle{危険なファイルの処理方法}
\end{frame}


\begin{frame}
  \frametitle{TeXliveの導入}
  \begin{enumerate}
    \item 「Install」を選択する。
    \item  「Next」を押して進み
    \item 「特定のミラーを選択」では日本のミラーサイトのどれでもよいので選ぶとよい
    \item インストール先は気にしなくてよい。
  \end{enumerate}
  

\end{frame}

\begin{frame}
  \frametitle{高度な設定}

  \begin{itemize}
    \item 「ディレクトリ」は無変更
    \item 「選択したもの」欄で変更2つ行う。
    \begin{itemize}
      \item スキーム(インストールするパッケージの種類を大まかに決める)を変更する。basic スキームを選択。
      \item 下のカスタマイズに入り、言語欄から日本語と英語を追加。右の他のコレクション欄から `LaTeX 推奨パッケージ' を追加。
    \end{itemize}
  \end{itemize}

  設定が完了したら右下のインストールを押す。
\end{frame}
\begin{frame}
  \frametitle{インストールの正常終了を確認する}
  インストールの後に、コマンドプロンプロトを起動する。
  コマンドプロンプトは、Windowsキーを押すと出てくるところから確認する。Mac の場合はターミナルから行う。ここに
  \begin{quotation}
    latex -v
  \end{quotation}
  と打ち込めばよい。

  Macでは、

\end{frame}


\begin{frame}
  \frametitle{なぜ\LaTeX を使うのか}
  %wordとの比較を交えつつ論文執筆のデファクトスタンダードとなっていること、数式の組版処理が優れていること、修正、再利用が容易なことなどを話す
  \LaTeX とは、文書執筆ツールとして使われる、組版処理系の一つである。
\LaTeX を使う利点をいくつか挙げる。
\begin{itemize}
  \item 数式がきれいに書ける
  \begin{itemize}
    \item Wordで書こうとするととんでもない数のクリックと精密なエイム力を要求される
    \footnote{これは決してWordを貶しているわけではない。そのようにWordを使うことが間違っているのである。Wordも\LaTeX も同じ文書執筆ツールではあるが、それぞれに長所短所があり使うべき場所というものがある。}
  \end{itemize}
  \item 見た目と論理構造を分離できる
  \item 修正、再利用が容易
  \item gitでバージョン管理できる
  \begin{itemize}
    \item 共同編集が可能
  \end{itemize}
  \item 数学系、物理系では論文執筆のデファクトスタンダード
  \item 貧弱なスペックのパソコンでも編集作業がやりやすい
  \item 無料
\end{itemize}
  
\end{frame}

\begin{frame}
  \frametitle{TeXlive Managerからパッケージのダウンロード}
  \begin{enumerate}
    \item WindowsのスタートメニューからTeXliveManagerを実行する。
    \item TeX live Managerを開くと現在TeX liveにあるパッケージが読み込まれる。
    \item そして、ここから検索欄に自分の必要なパッケージを読み込むと今必要なパッケージをダウンロードすることができる。
  \end{enumerate}
 検索欄から`latexmk'と検索する。すると、しばし待機した後に二つの`latexmk'が出てくる。
\begin{itemize}
  \item latexmk
  \item latexmk.win32
\end{itemize}
の二つが出てくるので、これらをチェックボックスをクリックして選択項目をインストール を選択し latexmk を選択してインストールする。これでlatexmkのインストールが完了する。

また今回は、\LuaLaTeX を用いるので、lualatex-mathというパッケージをダウンロードする。
\end{frame}

\begin{frame}
  \frametitle{VSCodeのインストール}

  Visual Studio Code は次のページからダウンロードするとよい。
\begin{quote}
	\url{https://code.visualstudio.com/download}
\end{quote}
をクリックしてそのあとの画面の中から今回の場合は、
自分の環境に合わせて (Windows か Mac の) ファイルをダウンロードする。
ファイルを開き、追加タスクの変更は触らないようにしてNextを押し続ければ
インストールできる。

\end{frame}

\begin{frame}
  \frametitle{latexmkの設定}
  次のコードをホームディレクトリ直下に入れる。
  
\end{frame}

\begin{frame}
  \frametitle{LaTeX workshopの設定}


\end{frame}

\begin{frame}
  \frametitle{Ultra Math Previewのインストール}
  

\end{frame}

\end{document}