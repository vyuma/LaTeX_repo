\documentclass{beamer}
\usetheme{CambridgeUS}       % テーマの選択
\usepackage{graphicx}        % 各種画像の張り込み
\usepackage{amsmath,amssymb} % 標準数式表現を拡大する
\usepackage{luatexja}        % 日本語対応
\usepackage[ipaex]{luatexja-preset} % IPAexフォントしたい

\setbeamertemplate{navigation symbols}{}      % ナビゲーションバー非表示
\usepackage{listings,jvlisting} %日本語のコメントアウトをする場合jvlisting(もしくはjlisting)が必要
%ここからソースコードの表示に関する設定
% \usepackage{lua-visual-debug}
\lstset{
	stringstyle={\ttfamily},
	commentstyle={\ttfamily},
	basicstyle={\ttfamily},
	columns=fixed,
  frame={tb},
  breaklines=true,
  columns=[l]{fullflexible},
	numbers=left,%行数を表示したければonにする
	numberstyle={\scriptsize},
  xrightmargin=0em,
  xleftmargin=3em,
  stepnumber=1,
  numbersep=1em,
	tabsize=2,
  lineskip=-0.5ex,
  backgroundcolor=\color{white}
}

\usepackage{bxtexlogo} % platex, lualatexのロゴを使うために必要

% \setbeamertemplate{theorems}[numbered]
% \addtobeamertemplate{theorem begin}{\normalfont}{}
% \uselanguage{japanese}
% \languagepath{japanese}
% \deftranslation[to=japanese]{Theorem}{定理}
% \deftranslation[to=japanese]{Corollary}{系}
% \deftranslation[to=japanese]{Lemma}{補題}
% \newtheorem{proposition}[theorem]{命題}
% \deftranslation[to=japanese]{Proposition}{命題}
% \deftranslation[to=japanese]{Example}{例}
% \deftranslation[to=japanese]{Examples}{例}
% \deftranslation[to=japanese]{Definition}{定義}
% \deftranslation[to=japanese]{Definitions}{定義}
% \deftranslation[to=japanese]{Problem}{問題}
% \deftranslation[to=japanese]{Solution}{解}
% \deftranslation[to=japanese]{Fact}{事実}
% \deftranslation[to=japanese]{Proof}{証明}
% \def\proofname{証明}

\title{VS Codeで\LaTeX を書く}
\subtitle{VS Codeの機能を使いこなす}
\author{リュカ, 裕磨}
\institute{ゼロイチゼミ, 学術サーバー}
\date{\today}

\begin{document}

\begin{frame}
  \titlepage
\end{frame}

\section{TeX Liveのインストール}
\subsection{TeX Live インストール}

\begin{frame}
  \frametitle{TeX Liveのインストール}
  \begin{enumerate}
    \item 次のページにアクセスする。

    \url{https://www.tug.org/texlive/acquire-netinstall.html}

    \item ページ上のリンクinstall-tl-windows.exeをクリックし、TeX Liveのインストーラーをダウンロードする。
    \item ダウンロードを終えたら実行する。
  \end{enumerate}
\end{frame}


\begin{frame}
  \frametitle{TeX Liveのインストール}
  \begin{enumerate}
    \item 「Install」を選択し、「Next >」を押す。
    \item 「Install」を押す。
    \item 「TeX Live 2024インストーラ」と書かれたウィンドウが出てくるまで待つ。
    \footnote{この際「特定のミラーを選択」で日本のサーバーを指定しておくとダウンロードが速く終わるかもしれない。}
    \item 特に設定は変更せず「インストール」を押す。
  \end{enumerate}
\end{frame}

% \begin{frame}
%   \frametitle{高度な設定}

%   \begin{itemize}
%     \item 「ディレクトリ」は無変更
%     \item 「選択したもの」欄で変更2つ行う。
%     \begin{itemize}
%       \item スキーム(インストールするパッケージの種類を大まかに決める)を変更する。basic スキームを選択。
%       \item 下のカスタマイズに入り、言語欄から日本語と英語を追加。右の他のコレクション欄から `LaTeX 推奨パッケージ' を追加。
%     \end{itemize}
%   \end{itemize}

%   設定が完了したら右下のインストールを押す。
% \end{frame}

\subsection{インストール完了の確認}
\begin{frame}
  \frametitle{インストールの正常終了を確認する}
  インストールの後に、コマンドプロンプロトを起動する。
  コマンドプロンプトは、Windowsキーを押すと出てくるところから確認する。Mac の場合はターミナルから行う。ここに
  \begin{quotation}
    latex -v
  \end{quotation}
  と打ち込めばよい。

\end{frame}

\section{\LaTeX とは}
\subsection{\LaTeX をなぜ使うのか?}
\begin{frame}
  \frametitle{なぜ\LaTeX を使うのか}
  %wordとの比較を交えつつ論文執筆のデファクトスタンダードとなっていること、数式の組版処理が優れていること、修正、再利用が容易なことなどを話す
  \LaTeX とは、文書執筆ツールとして使われる、組版処理系の一つである。
\LaTeX を使う利点をいくつか挙げる。
\begin{itemize}
  \item 数式がきれいに書ける
  \begin{itemize}
    \item Wordで書こうとするととんでもない数のクリックと精密なエイム力を要求される
    \footnote{これは決してWordを貶しているわけではない。そのようにWordを使うことが間違っているのである。Wordも\LaTeX も同じ文書執筆ツールではあるが、それぞれに長所短所があり使うべき場所というものがある。}
  \end{itemize}
  \item 見た目と論理構造を分離できる
  \item 修正、再利用が容易
  \item gitでバージョン管理できる
  \begin{itemize}
    \item 共同編集が可能
  \end{itemize}
  \item 数学系、物理系では論文執筆のデファクトスタンダード
  \item 貧弱なスペックのパソコンでも編集作業がやりやすい
  \item 無料
\end{itemize}
  
\end{frame}

\section{\LaTeX 環境構築}
\subsection{TeX Liveの設定}
\begin{frame}
  \frametitle{TeX Live Managerからパッケージのダウンロード}
  \begin{enumerate}
    \item WindowsのスタートメニューからTeX LiveManagerを実行する。
    \item TeX Live Managerを開くと現在TeX Liveにあるパッケージが読み込まれる。
    \item そして、ここから検索欄に自分の必要なパッケージを読み込むと今必要なパッケージをダウンロードすることができる。
  \end{enumerate}
 検索欄から`latexmk'と検索する。すると、しばし待機した後に二つの`latexmk'が出てくる。
\begin{itemize}
  \item latexmk
  \item latexmk.win32
\end{itemize}
の二つが出てくるので、これらをチェックボックスをクリックして選択項目をインストール を選択し latexmk を選択してインストールする。これでlatexmkのインストールが完了する。

また今回は、\LuaLaTeX を用いるので、lualatex-mathというパッケージをダウンロードする。
\end{frame}

\subsection{VSCodeのインストール}
\begin{frame}
  \frametitle{VSCodeのインストール}

  Visual Studio Code は次のページからダウンロードするとよい。

	\url{https://code.visualstudio.com/download}

  をクリックしてそのあとの画面の中から今回の場合は、
  自分の環境に合わせて (Windows か Mac の) ファイルをダウンロードする。
  ファイルを開き、追加タスクの変更は触らないようにしてNextを押し続ければ
  インストールできる。
\end{frame}

\subsection{VSCodeの設定}

\begin{frame}
  \frametitle{VS codeを日本語化する}
  VS Codeをショートカットから開いて、積み木のようなアイコンExtentions(拡張機能)
  をクリック。検索窓にJapaneseと打って、
  Japanese Language Pack for Visual Studio Code
  を選択してインストールする。これで日本語化が完了する。
  このように拡張機能を入れることでVS Codeの機能をより使いやすいものにすることができる。
  
  日本語化されない場合には、VSCodeを再起動すると良い。または、コマンドパレットを出現させる。これはCtrl+Shift+Pを押すと出現する。ここで、language とうって表示言語を日本語(Japanese)にすることでも日本語化することができる。

\end{frame}

\begin{frame}[fragile]
  \frametitle{\LaTeX を打つ}

  適当なファイルを生成する。そこに.tex ファイルを作成する。そこに下のコードを入れる。
  \begin{lstlisting}[caption=HelloWorld,label=HelloWorld]
  \documentclass{ltjsarticle}
  \begin{document}

  \title{はじめての\LaTeX }
  \author{Meidai}
  \maketitle
    \section{はじめての\LaTeX Lua\LaTeX }
      \subsection{小見出し!}
      Hello world!
      今日は\LaTeX を覚えていってください。
      \LaTeX + VSCode は最強の組み合わせ。
  \end{document} 
  \end{lstlisting}

\end{frame}


\begin{frame}[fragile]
  \frametitle{\LaTeX のコンパイルの仕方}
今作成した.tex ファイルをコンパイルする。コンパイルとは、\LaTeX のソースコードを実際の文書に変換することである。
\begin{enumerate}
  \item コマンドプロンプトを起動する。
  \item 作成した.tex ファイルがあるディレクトリに移動する。\\
  \verb|cd <ディレクトリへの相対パス>|
  \item 以下のコマンドを打ち込む。\\
  \verb|lualatex helloworld.tex|
\end{enumerate}
  
\end{frame}

\begin{frame}
  \frametitle{latexmkとは何か?}
  
  今回は一度ですんだが、相互参照などをするときには、コンパイルを複数回行う必要がある。
  手動でのコンパイルは手間がかかるため、latexmkを用いる。
  
  latexmkは、各種\LaTeX のビルドコマンドを自動で一括で実行してくれるツールである。

\end{frame}

\begin{frame}[fragile]
  \frametitle{latexmkの設定}

  \begin{enumerate}
    \item ホームディレクトリ
    \footnote{
      PC>ローカルディスク>ユーザー>ユーザー名、と辿れるディレクトリのこと}
    直下に.latexmkrcというテキストファイルを作成する。VSCodeで開く。
    \item 次のコードを入力する。
  \end{enumerate}
  \begin{lstlisting}[caption=latexmkの設定,label=code:latexmk]
    # LuaLaTeX のビルドコマンド
    $lualatex = 'lualatex %O -synctex=1 -interaction=nonstopmode %S';
    # Biber, BibTeX のビルドコマンド
    $biber = 'biber %O --bblencoding=utf8 -u -U --output_safechars %B';
    $bibtex = 'pbibtex %O %B';
    # makeindex のビルドコマンド
    $makeindex = 'upmendex %O -o %D %S';
    
    $pdf_mode = 4;
    \end{lstlisting}

\end{frame}

\begin{frame}
  \frametitle{LaTeX workshopのインストール}

  LaTeXworkshopとは、VSCode の拡張である。VSCode上で\LaTeX を使ううえで必須の拡張である。

  \begin{enumerate}
    \item 左の拡張機能アイコンをクリックする。
    \item 拡張機能の検索欄からlatexworkshopと入力する。
    \item latexworkshopを選び拡張機能をダウンロードする。
  \end{enumerate}
  latexworkshopの設定は、setting.json に記述する。
  VS Codeの設定は、setting.json
  というファイルに記載されている。

\end{frame}
\begin{frame}
  \frametitle{setting.jsonについて}
  setting.json を開く方法はいくつかある。
  \begin{description}
    \item[1つ目] ~\\
      VS Code左下の設定マーク(歯車マーク)をクリックして、「設定」を選択する。
      右上端にあるファイルに矢印がついたアイコンをクリックする。
    \item[2つ目] ~\\
      キーボードのショートカットキーを用いて、ctrl+shift+P と入力することで、
      コマンドパレットを出現させてそこにPreferences:Open User Setting
      と打ち込む。
    \item[3つ目]~\\
      Ctrl+ , で設定画面を開く。さらに右上端にあるファイルに矢印がついたアイコンをクリックする。
  \end{description}

\end{frame}

\begin{frame}[containsverbatim]
  \frametitle{setting.jsonの記述}
  setting.json を開いたら、\verb|{}|の中にmarkdownファイルに記載されたLaTeX workshopの設定をコピーして書き加え、保存する。
  同じ内容を下のソースコードに示す。
  もし開いた時点で\verb|{}|以外に何か書き込まれていた場合、\verb|{}|の最後の要素にコンマをつけて、設定をコピーして書き加え、保存する。


\end{frame}

\begin{frame}
  \frametitle{Ultra Math Previewのインストール}
Ultra Math Previewは、LaTeX Workshopよりも強力な数式プレビューができる拡張機能である。
パッケージで定義されたコマンドもPreview することができる。
さらにユーザー定義のmathPreview を導入することができるためにLaTeX Workshop 標準のプレビューよりもより利便性が高いものになっている。このmathPreview はMarkdownでも用いることができる。
\begin{itemize}
  \item 数式を打つとプレビューが即座に出てくる
  \item LaTeX と Markdown で同じマクロがプレビューに使える
  \item プレビューが透過できる
  \item プレビューの上からその下にある文字をクリックすることができる
\end{itemize}

拡張機能Ultra Math Previewは、Ctrl+Shift+Xで出てくる拡張機能の検索欄にUltra Math Preveiwとうち選択してインストールを押すことでインストールできる。

\end{frame}

\begin{frame}
  \frametitle{Ultra Math Previewの設定}

  設定方法は、以下の設定を入れれば設定される。

\end{frame}

\end{document}
