\documentclass{ltjsarticle}
% ltjsarticle: lualatex 用の 日本語 documentclass
% 他のタイプセットエンジンを使ってビルドする場合は,
%  \documentclass[dvipdfmx]{jsarticle} などとする.

\usepackage{url}
\usepackage[hidelinks]{hyperref}
\begin{document}

\title{最強の\LaTeX 環境構築 Windows版}
\author{yuma}
\date{2023/01/09}
\maketitle
\section{序文}
曖昧さを排除したできるだけ再現性の高い\LaTeX 環境構築
に対する処方箋を書きたいという思いでこの資料を書いた.
しかし環境構築後にも難関が待ち受けている.
そのために環境構築後の設定の変更等にも配慮して資料を工夫した.
\begin{itemize}
  \item \LaTeX を使えるようになりたい.さらに高性能エディタで書きたい
  \item 自分がものすごく苦労した環境構築を1日でかつ理解しながら終わらせたい.
  \item さらに,その先を自分で学習できるようにしたい.
  \item 苦労した環境構築の記憶をとどめておいてまた困った時に見返したい.
  \item 新大学1年生にも分かりやすく伝えたい.
\end{itemize}
という目標をもって書いた.同じような人の助けになれば幸いである.
可能な限り初心者に分かりやすくかつ自分が忘れないように書いたつもりである.
\LaTeX の世界は広大すぎて分からないがwordよりも数式が美しく書ける.
それだけではなく,論文は圧倒的に\LaTeX のほうが書きやすいといわれている.
この文章は,\LaTeX をVisual Stadio Codeでの環境構築のために自分が
したことのすべてを書いている.技術的解説もできるだけするように努めた.
実は\LaTeX の環境構築方法を完全に技術敵側面にも触れてしっかりVScode上
に入れるための方法を体系的にまとめた資料はネットには
存在しない可能性がある。もちろん分かりやすくまとめたサイトはたくさんあるが
玉石混交である。そのために、VScode 設定と、LaTeX 設定の両面のアプローチ
から環境構築設定を解説したいと思った。


この資料がどのようにほかの文書と違うのかを明確にする.この資料は,\LaTeX
の解説書ではない.\LaTeX をVScode上で動かすために必要なIT知識と
設定の方法を述べる.そのあとVSCodeの基本機能について説明する.
美文書書作成入門\cite{美文書本}ではVSCode上での環境構築や
何をすべきかが分からない.またほかのサイトなどを探しても,IT知識が
少しでもないと全く分からない.そのためにシステムをいじりたくなくなるし,
また,取り返しのつかなくなることがまれにある.この状況を回避するために
この資料を作った.


この資料は,すべて\LaTeX によって書いた.見にくい部分もあるかもしれないが
これは,著者の技術によるところもあるので許してほしい.
VScodeとLateXを同時に使うためには、LaTeXの書き方の知識、VSCodeの知識、
Windows に関する知識、LaTeX workshopに関するさらには、.jsonファイル
についての知識が必要になる。
このsetting.json をいじるためには、Java script の基本的な書き方の規則を
知らなければならない。ゆえに初心者には環境構築は難し過ぎる。
しかしそれを解決するための資料にしたいという編集の基本方針にする。


しかし,この資料にも至らぬ点が山ほどあると思われる.
自分もすべてを理解しているわけではない.
この資料の記述は自分がおこなってできたことだけを記述している.
それ以外のことを知りたい場合は,

\subsection*{追記事項}
さらに便利な,SyncTeX の機能を追加するための章を加えた.
VSCodeの特徴と拡張機能についての章を追加した.
LaTeX workshop のエラーについての章を追加した.
setting.json等のコード表を.md ファイルを用いて
配布する形式に変更した。このために、配布ファイルは2つになる。

varsion管理  ver2.0
\tableofcontents

\section{\LaTeX  VSCode とは何か?}
\subsection{\TeX とは何か?}
\TeX とは,組版ソフトウェアのことである.これを使うと数式がきれいにかけたり,
自動で組版をしてくれたりする機能がある.
詳しくは,美文書書作成入門\cite{美文書本}に詳しい.
ここで、\TeX と\LaTeX は全く違うものであることは記憶に値する。
この資料では,VSCode とLaTeX で組版を行うが,LaTeX works editer やcloud LaTeX 
というものもあり,最初はそちらを使うとVScodeの威力がわかると思う.
\subsection{VSCode とは何か?}
VSCodeとは,Visual Stadio Code の略でマイクロソフト社が提供する
統合開発環境である.そしてエディタとしても高性能である.
VScodeは,
\begin{itemize}
  \item 無料で頒布されている.
  \item 動作が軽い
  \item クロスプラットフォームである.(様々なOSに対応している.
  \item 最新トレンド全部入り
  \item 拡張がしやすく,様々な機能がある.
\end{itemize}
という特徴がある.
\section{\TeX に慣れる.}
美文書作成入門\cite{美文書本}や一週間で\LaTeX の基礎礎が学べる本
\cite{一週間基礎}等を用いて,\LaTeX にふれることで
この後の流れが少し分かりやすくなるかもしれない。
そこで、環境構築せずにも\TeX が使える環境としてcloud \LaTeX 
について解説する。
\subsection{cloud \LaTeX の使用}
cloud \LaTeX は、オンライン上で\LaTeX を扱うことが
できる。環境構築を設定する必要なくすぐにLaTeX 文章を書くことができるために
TeXに慣れるという面では、有用である。したがってこの章では、cloud LaTeX の
使い方について簡単に説明して\TeX に慣れてもらいたい。
\subsection{cloud \LaTeX の利用}
初めに
\url{https://cloudlatex.io/ja}
のサイトに行きユーザー名やメールアドレスと任意のパスワードを設定する。
すると設定したメールアドレスにメールが来るので、メールアドレスの受信確認
をクリックする。アカウント登録が完了する。
マイページから、新規プロジェクトの追加を選んで\LaTeX を書き始めることが
できる。

\section{ローカル環境構築の基本用語}
\subsection{\LaTeX 環境設定用IT用語の理解}

\begin{itemize}
  \item 環境 \LaTeX が機能するための設定やハードソフト等の存在などを
        合わせて環境という.
  \item ユーザー 自分が設定した名前 自分のこと
        ユーザーの名前のフォルダーが最上位にあると考えておけばよい.
  \item デフォルト 初期設定のこと。
  \item 隠しファイル エクスプローラー(よくファイルを使う場所)
        からは見えないファイル.例えば,.latex 等の.の前に
        何もついていないファイルのこと.
        開き方 エクスプローラーを開いて上にある表示タブ
        を開いて表示タブにする.表示タブから隠しファイルを開いて,
        チェックマークをつける.これにより,
        今まで見えなかったファイルが見えるようになる.

  \item ディレクトリ ファイルが入っている階層を指し示す言葉。
        階層のことであり、
        日本語の意味では住所録の意味.例えば,ドライブ直下に置くとは,
        最上部のフォルダーの中に入れることを意味する.
        ファイルの現在位置を指し示す言葉で、ルートディレクトリとは、
        フォルダーの階層の最上位のフォルダーを意味する。


        ここで注意! LaTeXや,その他のプログラミング言語は
        日本語つまり全角のファイル名は,探すことができない.
        エラーが出てしまう.このため,すべてのファイル名や,
        パソコンでの設定した名前は,かならず半角英数字にすること
        を覚えておく必要がある.
  \item パス ディレクトリでそのファイルの住所位置を
        指定するもの.
  \item プロジェクトルート ディレクトリの一番初めに存在する
        それより上のファイルが存在しない領域
  \item グローバル すべてのユーザーやそのアプリケーション内ですべてに設定
        されるような設定のこと。
  \item コマンドプロンプト Windowsの操作をコマンドでおこなうためのシステム.\\
        開き方1 コマンドプロンプトをスタートメニュー(Windowsアイコンのタブ)
        の中の検索窓からコマンドプロンプトと打ち,マウスで出てきた
        コマンドプロンプトをクリックする.\\
        開き方2 Windowsキー+R で「ファイル名を指定して実行」という窓が出てくる
        のでその指定する窓にCmd と打ち Enterを打つことで,
        コマンドプロンプトを実行できる.
  \item リポジトリ
        プロジェクトを構成するプログラムのソースコードやドキュメント、
         関連する各種のデータやファイルなどを一元的に管理する格納場所のこと

\end{itemize}

\subsection{\LaTeX 用語の理解}
\begin{itemize}
  \item ビルド \LaTeX では,記述したソースコードに問題がない
        かどうかの解析を行い,問題がなければ実行可能なファイルに変換すること.
  \item コンパイラ コンパイラとは機械が読み取れる言語に移すことのできるソフトウェア等のこと.
  \item コンパイル ソースコードを機械が翻訳できる言語に移すことで,ビルドの中の一連の作業に入っている.
  \item プリアンブル \verb|\documentclass|から\verb|\begin{document}|までの中にある設定のこと.
  \item コーディング コードを各作業のこと.
\end{itemize}


\section{\LaTeX VScode 導入篇}

\subsection{\LaTeX 導入}
\LaTeX は,次のページからダウンロードすると良い.\\
\url{https://www.tug.org/texlive/acquire-netinstall.html}\\
をクリックして,ページ上のリンク install-tl-windows.exeをクリックする.この時,
警告が出るが無視して大丈夫.
そしてNextを押し続けてinstallを押す.これで,\LaTeX packageのほとんどすべてを
ダウンロードすることが可能.
\subsubsection*{正常なインストール完了を確認する.}
インストール完了後,コマンドプロンプトを起動して,latex と打ち込む.
この時に,正式なversion情報が出てこれば正常に完了している.Versionは,$3.141592653$
のように円周率に近づいている.\\
(補足 これは,\TeX 製作者Donald E. Knuth教授の意向で
version up ごとに円周率$\pi$に近づいて行く.)
\subsection{Visual Stadio Code 導入}
Visual Studio Code は次のページからダウンロードするとよい.\\
\url{https://code.visualstudio.com/download}\\
をクリックしてそのあとの画面の中から今回の場合は,
Windows を選択してダウンロードする.
ファイルを開き,追加タスクの変更は触らないようにしてNextを押し続ければ
installできる.
VSCodeをアイコンから開いて,積み木のようなアイコンExtentions(拡張機能)
をクリック.検索窓にJapaneseと打って,
Japanese Language Pack for Visual Studio Code
を選択してインストールする.これで日本語化が完了する.
これにより,VScode側の基本設定は終了.
\section{環境構築篇}
環境構築の手順を順を追って紹介する.
この手法であれば今の環境下では大丈夫
\subsection{エクスプローラーの構築}
隠しファイルが見えるようにする.
開き方 エクスプローラーを開いて上にある表示タブ
を開いて表示タブにする.表示タブから隠しファイルを開いて,
チェックマークをつける.これにより,
今まで見えなかったファイルが見えるようになる.

\subsection{.latexmkの構築} % (fold)

\label{sub:.latexmkの構築}
% subsection .latexmkの構築 (end).latexmkの構築
隠しファイルを開いてディレクトリをユーザーディレクトリの直下におく.
このファイルはメモ帳でおこなってもよいし,VScode上でおこなってもよい.
ない場合は新たに作ること.エクスプローラーのWindowsフォルダーの
中のユーザーフォルダーの中に自分の名前が入っているフォルダがある.
ここに次のコードを制作した名前が.latekmkという名前のファイルを
保存する.つまりルートフォルダにこのファイルを保存するということ.
latexmkとは,各種latexのビルドコマンドの実行を適切な順序で
適切な回数おこなうためのツールです.


この作業が終わったら,次はVScodeを起動します.
\subsection{setting.jsonを開く}
setting.json を開くためには2通りある.\\
1つ目\\
VScode左下の設定マーク(歯車マーク)から一番上のタブのコマンドパレット
から,Preferences:Open User Setting と打ちこむ方法\\
2つ目\\
キーボードのショートカットキーを用いて,ctrl+shift+P と入力することで,
コマンドパレットを出現させ,そこにPreferences:Open User Setting
と打ち込む方法\\
の2つの方法があり,熟練したら2番目の方法の方が効率が良い.
\subsection{setting.jsonの設定}
.json はジェイソンと呼びます.
.json はJavascript に対応しています.
setting.json を開いたら,そこに何か書き込まれていた場合それには触らずに,
\verb|{}|
がかならずあるので,その中に次のコードをコピーして張り付ける.

\subsection{platexのコンパイルについて}
platexのコンパイルについて
platexに関する設定だけは,うまくいかないので下のように書き換える.
platex のところの
recipe tools の変更をおこなう.
setting.jsonのlatex-workshop.latex.toolsのところに次のように書き換える.

と書き換えれば良い.
\subsection{setting.json の各種機能}
\begin{verbatim}
  "latex-workshop.latex.autoBuild.run": "never",
\end{verbatim}
は,自動コンパイル機能である.しかし,自動コンパイルは,
このファイルのように大きすぎると非常に重くなってしまう.
\LaTeX は,コンパイルと,書くことを分離することで,
書く必要量をへらしているために,
これはneverにしておくことをおすすめする.


\section{\LaTeX の基本機能の設定}
\LaTeX の基本精神は,人間は,内容に集中して \LaTeX はデザインに
集中するという使い方をすること.余計な装飾はできるだけ避けるのが無難.
\subsection{コンパイル}
\begin{itemize}
  \item ctrl+Alt+Bのショートカットキーでおこなう.
  \item VSCode右上の緑色の三角マークをクリックして行う.
  \item 左のタブのBuild LaTeX project の中のRecipe:からおこなう.
\end{itemize}
ただし,今回の設定では,上の2つの方法でコンパイルするとlua\LaTeX
で出力される.このために,pLaTeXなどを使いたい場合は,
左のタブからおこなう必要がある.
\subsection{Latex indent}
LaTeXのソースを自動で成形してくれるツールこれは、ソースに自動で空白等を
入れてくれる環境がTeXliveにデフォルトで入っている。
latex indent はVScode上で,shift+Alt+Fをするとできる.
これはitemize環境等に空白を入れて見やすくする機能がある.
\subsection{\LaTeX で書いてみる.}
次の\LaTeX の文章を入力する.これは,lua \LaTeX で書くとよい.


\section{エラーメッセージを読む}
LaTeX workshop ではエラーがメッセージで表示されるがそれを
初心者はなかなかうまく使いこなせない.
そこでエラーの読み方と対応策を考える.
\subsection{Recipe terminated with error}
おそらくLaTeX Workshop を使って初めて見るだろうエラー.
このエラーは,LaTeX側のエラーではない.LaTeXのエラーは,
Ctrl+Shift+M のショートカットキーで表示される.
これは,注意が出ている時にも使うことができる.
\subsection{Recipe terminated with error. Retry building the project.}
このエラーというよりかは警告なのだがこれも本質的に意味がない.
\subsection{意味のあるエラーメッセージ}
意味のあるエラーメッセージは,出力される'問題'のログかまたは,"*.log"
というLaTeXのコンパイルの際に使われる中間生成物のファイルである.
今回のデフォルト設定では,このファイルは削除されるため出てこないが,
エラーが特定しやすくするためには,残しておいた方が良い可能性がある.

\section{新規packageの導入}
新規パッケージつまり,インターネット上や,
自分で作ったスタイルファイル等を使いたいときに,めんどくさいことになる.
パッケージファイルの拡張子は “.sty” であり,
このパッケージを適切な場所に配置しなければならない.
しかし,".sty"が入っていない場合がある.これに対処するためには次セクション
\subsection{".sty"ファイルが入っていない場合}
".ins"ファイルが重要になる.
このファイル形式は,パッケージ本体と".ins"ファイルが入っている.dtxを
LaTeXで.dtx ファイルを実行する必要がある.\\
例えばtools.ins というファイルがあった場合には,.dtx ファイルを
同じディレクトリに保存する.次に
コマンドプロンプトを起動して,以下のように,入力する.
\begin{verbatim}
  latex tools.ins
\end{verbatim}
すると,同じフォルダ内に.sty ファイルが生成される.
この.styファイルの保存場所は,次セクション参照.
\subsection{TeX Directory Structure とは何か?}
texのパッケージなどは,TDS(TeX Directory Structure)にしたがって各
ディレクトリに配置されている.
今回はtexliveなので,エクスプローラーからwindowsフォルダーの中の
texlive フォルダーの中の2022年フォルダーの中のtexmf-distの中のtexの中の
latexフォルダーの中に各パッケージのフォルダーがあり,
その中に各スタイルファイルフォルダーが保存されている.といった構成に
なっている.
これは非常に見にくいし,読みにくいので,ここからは,これを
\begin{verbatim}
  C:\texlive\2022\texmf-dist\tex\latex
\end{verbatim}
のように書くと簡便になる.
そしてこれがWindowsのディレクトの記述の仕方である.
この中のtexフォルダーにはたくさんのフォルダーがあるが,
最初に覚えておくべきは2つで,
\begin{center}
  \begin{tabular}{lrr} \hline
    folder名 & 意味                               \\ \hline
    tex     & TeXの操作に関するフォルダーで,latex フォルダーがある. \\
    font    & フォントにかかわるフォルダー.                  \\ \hline
  \end{tabular}
\end{center}
であり,このことから,.styファイルの保存場所は,
\begin{verbatim}
  C:\texlive\2022\texmf-dist\tex\latex
\end{verbatim}
よりも下のファイルに加えれば良いことがわかった.
フォントファイルについては,以下参照.
\url{https://texwiki.texjp.org/?TeX%20%E3%81%AE%E3%83%87%E3%82%A3%E3%83%AC%E3%82%AF%E3%83%88%E3%83%AA%E6%A7%8B%E6%88%90}
\subsection{パッケージを使うための一覧表の更新}
texlive環境では,上のように.styファイルを保存しても動かない.これは,
ls-Rというファイルが存在していてこれが,TeXが必要なパッケージを探すための
パスを与えている.そのために,この一覧表を更新しなければならない.
このためには,コマンドプロンプトを開き,次のように命令するだけでいい.
\begin{verbatim}
  mktexlsr
\end{verbatim}
と入力することで一覧表を更新することができる.
\subsection{使用するべきpackage}
packageは最小構成でおこなうべきである。
なぜなら、lualatex  はかなり最近のものであり、packageがlualatexに
対応していないということもかなりの可能性としてある。
そのため、参考になるのは参考文献のサイト
\section{VScodeの基本機能}
\subsection{、。を,.にする<置換機能>}
VScodeの基本機能として,置換機能がついている.このためマクロを組まずとも
,を,に変えることができる.
コマンドは,Ctrl+Hキーを押すと検索と置換が出てきて,Enter キーを押すと
ひとつづつ置換,Crl+Alt+Enter キーを押すとすべてが置換される.
また置換を閉じるのはEsc キーである.
\subsection{コメント機能}
\LaTeX のコメントは%である.しかしこれをいちいち書くのは面倒くさい.
さらにプログラム言語によりコメント機能は全く違う.この時に,Ctrl+/ を
使うことで,どんなプログラム言語にも対応したコメント機能が
すぐにつけられる
\subsection{SyncTexを使う}
\subsubsection*{SyncTex とは何か?}
TeX のソースファイルと PDF でカーソルの位置を同期する機能のこと.
これを使えば,pdf上の表示がどのソースに関係しているかということが
すぐにわかる.
\subsubsection*{SyncTeXの設定}
SyncTeXとは,ビルドするときに形成されるファイル
によって動かすことができる.つまりファイルが必要になる.
そのファイルの名前は,'.synctex.gz'である.
しかし,今のsetting.jsonでは,
このファイルは消去される設定になっている.
ゆえに,このファイルを消去しないようにしなければならない.
したがって,setting.jsonから
\begin{verbatim}
  "latex-workshop.latex.clean.fileTypes":
\end{verbatim}
の記述の中から,
\begin{verbatim}
  "*.synctex.gz",
\end{verbatim}
の記述を消去すればよい.これにより,VSCodeのpdfvierにおいて,
SyncTeXが利用できるようになる.
\subsubsection*{SyncTeXの具体的な利用の仕方}
pdfviewer上で,Ctrl キーを押しながら,マウスでコードをみたい場所におき,
左クリックすることで,そのコードの位置に飛ぶことができる.
逆にコードからpdfに飛びたいときは,選択範囲をマウスで示して,または
カーソルを示して, Ctrl+Alt+J で飛ぶことができる.
(コマンドパレットからSyncTeXと入れてコマンドを実行させてもよい.)

\subsection{各種コマンド}
\begin{center}
  \begin{tabular}{lrr} \hline
    コマンド            & 機能                  \\ \hline
    Ctrl+Alt+B          & ビルドを実行する.           \\
    Ctrl+Alt+V          & pdfviewer を起動する.    \\
    Ctrl+click          & SyncTeXの利用(pdf側)    \\
    Ctrl+Alt+J          & SyncTeXの利用(コード側)    \\
    Ctrl+Hキー            & 置換パレットの表示           \\
    Crl+Alt+Enter       & 置換の全置換              \\
    Ctrl+/              & 行のコメントアウト機能(全コード共通) \\
    Ctrl+@              & VScodeのターミナルの起動     \\
    Shift+Alt+F         & LaTeXindentを実行する.\\
    Ctrl+Space          & コード補完を再表示\\
    Ctrl+Shift+'+ or -' & ズームの程度を調整する.        \\ \hline
  \end{tabular}
\end{center}
ただし、ビルドコマンドはビルドをするのは、Build LaTeX project の最初に
設定されているコマンドがデフォルトで設定されている。
\subsection{スニペット導入}
\subsubsection*{スニペットとは}
スニペットとは,コードの中で何回も使うだろうコードを少しの記述
でそれを呼び出す一連の動作のことをいう.スニペットの語源は,
短い単語のことである.
\subsubsection*{スニペットの導入}
まずスニペットを導入する.左下の歯車から「ユーザースニペットの構成」を
選択する.検索窓が出てくるので,latexと入力して latex.json を開く.
latex.json にも,\verb|{}|があるので,その中に次のコードをコピーする.
スニペットの導入は,setting.jsonとは異なるため環境を破壊することはない.


と書く.例えば,report とlatexで打つと,
reportのひな形が出てくるようになる.
このように,スニペットはプリアンブル部等
を簡単に早く書くことができるようになる.
\subsection{スニペットの書き方}
スニペットの自作フォーマットは次のように書く.
スニペットもsetting.jsonと同じように,','で各設定を区切る.
\begin{verbatim}[caption=latex.json,]
	{
  "[ スニペットの名前 ]": {
    "prefix": "[ 呼び出すときのショートカット]",
    "body":[
      "[ 出力されるコードの1行目]",
      "[ 出力されるコードの2行目]",
      "...",
    ],
    "description": "[ スニペットの説明文]"
  }
  }
\end{verbatim}
のように書く.
\subsubsection*{入力値の補足}
入力値の中のコマンドに\verb|$n|というのがあるがこれは,スニペットを
記述するときに入力するためのカーソルが
次にどこにいけば良いかを入力するための引数である.
例えば,次の例では,プログラミングを載せるための環境をスニペットにより
定義している.
\begin{verbatim}[caption=latex.json,]
	{
  "[ スニペットの名前 ]": {
    "prefix": "[ 呼び出すときのショートカット]",
    "body":[
      "[ 出力されるコードの1行目 $1]",
      "[ 出力されるコードの2行目 $2]",
      "...",
      "$1"
      "",
    ],
    "description": "[ スニペットの説明文 ]"
  }
  }
\end{verbatim}

このように書いたときに,カーソルが\verb|$1|を記述した後にTabキーを押すと
\verb|$2|にカーソルが移動するように書くことができる.
また,さらに,\verb|$1|が2つあるが,これらは,同じことを記述したい場合,
例えば参照をタイトルと同じにしたい場合に両方一度に入力することができる.
また,bodyの末尾に""を入れておけば,入力終了後に
次の行からすぐに書き始めることができる.


\section{VSCodeの拡張機能}
VSCodeの拡張機能は,できるだけ入れないことが基本.VSCodeの基本機能
として使える機能があればわざわざ拡張機能を入れるべきではない.
これは,VSCodeの特徴セある「軽い」という特徴を阻害するからである.
しかし入れるべき拡張機能はもちろん存在する.
\begin{itemize}
\item LaTeX workshop 基本設定で使う。
\item cloud LaTeX 連携
\item CaTeX
\item LaTeX Utilities
\end{itemize}
\subsection{\LaTeX workshop の詳細設定}

\subsection{cloud\LaTeX との連携}
\subsection{CaTeX を導入する.}
CaTeX1(軽鳥/怪鳥)とは,Hiromi ISHII が作った,LaTeX workshop とともに使うことを
想定した,VSCode上での拡張機能です.
イメージ補完などが使える.
これは必要になったら導入するくらいで十分であると思う。
CaTeXを追加するとCtrl+C のコマンドが使えなくなるということが発生した。
これは、CaTeX が勝手にショートカットキーを作るためにコピーがうまく使えなくなる。
これを解決するためには、keyショートカットキーを変更しなければならない。
その方法の説明はほかの章に譲る。


\section{Git 連携}
VSCode にはGit というバージョン管理システムがあり、
これを用いることで巨大な文書制作でのバージョン管理が効率化するという
利点がある。
そしてGit はVScode上で開くことができる。
これはローカル環境にバージョン管理のためのレポジトリを生成できる。
これによりバージョン管理が非常に簡単になりさらにどこを変更したのかということが
非常に簡単にわかる。
したがって今までCtrl+z で前の状態に戻っていたと思うが、
セーブするとその前以上には戻ることができなかった状況から
これは理論的には無限の過去の状態に戻すことができる状態になる。
なおかつローカルにもオンライン(github)
として共同作業をオンラインですることができたり、また全世界に公開して
共同作業ができたりする。
このように複数人でおこなう文書やプログラミング制作で大いに役立つ機能である。
\subsection{Git の導入}
git はVSCodeの3つ目のタブにある丸が3つつながっているようなタブを
クリックするとGitの導入を促すようなものがあるのでそれをクリックして
Git のソフトウェアをGit のホームページからダウンロードする。
\subsection{github の導入}
github はgitと連携してネット上にレポジトリを生成することできる
\section*{付録}
\subsection*{ショートカットキー一覧}
\begin{center}
  \begin{tabular}{lrr} \hline
    コマンド            & 機能                  \\ \hline
    Ctrl+Alt+B          & ビルドを実行する.           \\
    Ctrl+Alt+V          & pdfviewer を起動する.    \\
    Ctrl+click          & SyncTeXの利用(pdf側)    \\
    Ctrl+Alt+J          & SyncTeXの利用(コード側)    \\
    Ctrl+Hキー            & 置換パレットの表示           \\
    Crl+Alt+Enter       & 置換の全置換              \\
    Ctrl+/              & 行のコメントアウト機能(全コード共通) \\
    Ctrl+@              & VScodeのターミナルの起動     \\
    Shift+Alt+F         & LaTeXindentを実行する.\\
    Ctrl+Space          & コード補完を再表示\\
    Ctrl+Shift+M        & \LaTeX のエラーメッセージを表示する\\
    Ctrl+shift+M & 数式環境で数式のプレビューを表示する。\\ 
    Ctrl+B              & サイドバーの表示の設定\\
    Ctrl+Shift+'+ or -' & ズームの程度を調整する.        \\
    Ctrl+K F            & ワークスペースを閉じる.\\ \hline
  \end{tabular}
\end{center}


次の参考文献は,
この資料を書き上げるために用いた資料一覧である.
この資料を読み通すことができたならば,
きっとこれらのサイトも読むことができるだろう.
さらに便利な使い方や,自分なりの設定をすることもできるだろう.
良い\LaTeX ライフを送れるように
\begin{thebibliography}{999}
  \bibitem{美文書本}
  奥村晴彦/黒木裕介(2020)
  \LaTeX 2e 美文書書作成入門
  \bibitem {完全導入ガイド}
  @passive-radio\\
  【大学生向け】LaTeX完全導入ガイド Windows編 (2022年)\\
  \url{https://qiita.com/passive-radio/items/623c9a35e86b6666b89e#4-snippet-%E3%81%AE%E3%82%B9%E3%82%B9%E3%83%A1%E6%96%87%E7%AB%A0%E4%BD%9C%E6%88%90%E3%81%AE%E3%82%82%E3%81%A3%E3%81%A8%E5%8A%B9%E7%8E%87%E5%8C%96}
  \bibitem{最高の環境latex}
  @rainbartown\\
  VSCode で最高の LaTeX 環境を作る(2020)
  \url{https://qiita.com/rainbartown/items/d7718f12d71e688f3573}
  \bibitem{platex}
  日記\\
  【Windows】VSCode+pLaTeX(+LuaLaTeX)環境を構築した\\
  \url{https://everykalax.hateblo.jp/entry/2022/12/15/144238}
  \bibitem{スニペット1}
  Web備忘録\\
  VSCode でスニペットを自作する方法(2019)\\
  \url{https://webbibouroku.com/Blog/Article/vscode-snippets}
  \bibitem{スニペット2}
  @michawo\\
  VSCodeで自作のスニペットを登録する\\
  \url{https://qiita.com/michawo/items/051da6ce6d9daf9784fb}
  \bibitem{スニペット完全解説}
  Web業界で働く人を少しだけ手助けするメディア\\
  Visual Studio Code ユーザースニペットの使い方まとめ\\
  \url{https://web-guided.com/620/#:~:text=%E3%82%B9%E3%83%8B%E3%83%9A%E3%83%83%E3%83%88%E3%82%92%E5%B1%95%E9%96%8B%E3%81%97%E3%81%9F%E6%99%82%E3%81%AB%E3%80%81%E3%82%AB%E3%83%BC%E3%82%BD%E3%83%AB%E3%82%92%E4%BB%BB%E6%84%8F%E3%81%AE%E4%BD%8D%E7%BD%AE%E3%81%AB%E8%A8%AD%E5%AE%9A%E3%81%99%E3%82%8B%E3%81%93%E3%81%A8%E3%81%8C%E5%87%BA%E6%9D%A5%E3%81%BE%E3%81%99%E3%80%82%20%241%20%E3%82%92%E8%A8%98%E8%BF%B0%E3%81%97%E3%81%9F%E7%AE%87%E6%89%80%E3%81%AB%E3%82%AB%E3%83%BC%E3%82%BD%E3%83%AB%E3%81%8C%E5%87%BA%E7%8F%BE%E3%81%97%E3%80%81%20%242%20%E3%80%81%20%243%20%E3%81%A8%E8%A8%98%E8%BF%B0%E3%82%92%E8%BF%BD%E5%8A%A0%E3%81%97%E3%81%9F%E5%A0%B4%E5%90%88%E3%81%AF%E3%80%81%20%EF%BC%BB,%EF%BC%BD%20%E3%82%AD%E3%83%BC%E3%82%92%E6%8A%BC%E3%81%99%E3%81%93%E3%81%A8%E3%81%A7%E3%80%81%E3%81%9D%E3%82%8C%E3%82%89%E3%81%AE%E4%BD%8D%E7%BD%AE%E3%81%AB%E3%82%AB%E3%83%BC%E3%82%BD%E3%83%AB%E3%81%8C%E7%A7%BB%E5%8B%95%E3%81%97%E3%81%BE%E3%81%99%E3%80%82%20%22body%22%3A%20%5B%20%22console.log%20%28%27%241%27%29%3B%22%2C%20%22%242%22%20%5D}
  \bibitem{ルートディレクトリ解説}
  大橋\\
  ルートディレクトリって結局どこ?\\
  https://codor.co.jp/django/root-directry
  \bibitem{TDS}
  TeX Wiki\\
  TeX のディレクトリ構成 (TDS) \\
  \url{https://texwiki.texjp.org/?TeX%20%E3%81%AE%E3%83%87%E3%82%A3%E3%83%AC%E3%82%AF%E3%83%88%E3%83%AA%E6%A7%8B%E6%88%90}
  \bibitem{コマンドライン}
  Django Girls のチュートリアル\\
  コマンドライン(コマンドプロンプト)とその関連について学べるサイト\\
  \url{https://tutorial.djangogirls.org/ja/}
  \bibitem{コマンドプロンプト}
  HP社\\
  Windows 10 のコマンドプロンプトとは?その起動方法と使用例を紹介\\
  https://jp.ext.hp.com/techdevice/windows10sc/27/
  \bibitem{ビルド}
  IT用語辞典\\
  ビルド【build】\\
  \url{https://e-words.jp/w/%E3%83%93%E3%83%AB%E3%83%89.html}
  \bibitem{ソースコード貼り付け}
  \verb|@ta_b0_|\\
  LaTeXにソースコードを【美しく】貼る方法\\
  \url{https://qiita.com/ta_b0_/items/2619d5927492edbb5b03}
  \bibitem{lualatexハイパーリンク}
  LuaLaTeX Lab\\
  【基本】LuaLaTeXのPDFを便利にしよう ~hyperrefパッケージほか~\\
  \url{https://lualatexlab.blog.fc2.com/blog-entry-43.html}
  \bibitem{各種コマンド}
  KERI's Lab\\
  VSCode で TeX を書こう\\
  \url{https://www.kerislab.jp/posts/2019-01-14-vscode-latex/#:~:text=SyncTex%20%E3%81%A8%E3%81%AF%E3%80%81TeX%20%E3%81%AE%E3%82%BD%E3%83%BC%E3%82%B9%E3%83%95%E3%82%A1%E3%82%A4%E3%83%AB%E3%81%A8%20PDF%20%E3%81%A7%E3%82%AB%E3%83%BC%E3%82%BD%E3%83%AB%E3%81%AE%E4%BD%8D%E7%BD%AE%E3%82%92%E5%90%8C%E6%9C%9F%E3%81%99%E3%82%8B%E6%A9%9F%E8%83%BD%E3%81%A7%E3%81%99%E3%80%82%20VSCode%20%E3%81%AE,PDF%20%E3%83%93%E3%83%A5%E3%83%BC%E3%83%AF%E3%81%AF%20SyncTex%20%E3%81%AB%E5%AF%BE%E5%BF%9C%E3%81%97%E3%81%A6%E3%81%84%E3%82%8B%E3%81%AE%E3%81%A7%E3%80%81TeX%20%E3%83%95%E3%82%A1%E3%82%A4%E3%83%AB%E3%81%A8%20PDF%20%E3%83%95%E3%82%A1%E3%82%A4%E3%83%AB%E3%81%AE%E8%A9%B2%E5%BD%93%E7%AE%87%E6%89%80%E3%82%92%E8%A1%8C%E3%81%8D%E6%9D%A5%E3%81%99%E3%82%8B%E3%81%93%E3%81%A8%E3%81%8C%E3%81%A7%E3%81%8D%E3%81%BE%E3%81%99%E3%80%82}
  \bibitem{SyncTeX}
  TeX Wiki\\
  SyncTeX\\
  \url{https://texwiki.texjp.org/?SyncTeX}
  \bibitem{LaTeX+VScode wiki}
  TeX Wiki\\
  Visual Studio Code/LaTeX\\
  \url{https://texwiki.texjp.org/?Visual%20Studio%20Code%2FLaTeX#he95e080}
  \bibitem{SyncTeX VSCode}
  TeXフォーラム\\
  VScodeでSyncTeXが使えない\\
  \url{https://okumuralab.org/tex/mod/forum/discuss.php?d=3075&parent=18242}
  \bibitem{SyncTeX+各種設定}
  \verb|@t_kemmochi|\\
  uplatexでSyncTeXするための最低限のVS Code設定\\
  \url{https://qiita.com/t_kemmochi/items/dd38bbf2b823c770d1ec}
  \bibitem{絶対パス設定ミス例}
  @yt1114\\
  SyncTex で2時間消費…VSCode で SyncTex を適用するときの注意点\\
  \url{https://qiita.com/yt1114/items/a50c97dafb4d193c0198}
  \bibitem{SyncTeX TeXwork}
  情報科学屋さんを目指す人のメモ\\
  PDFをクリックして対応するLaTeXソースにジャンプする方法(TeXworks+SyncTeX)\\
  \url{https://did2memo.net/2015/03/05/latex-pdf-source-jump/}

  \bibitem{一週間基礎}
  明松真司(2022)\\
  1週間でLaTexの基礎が学べる本
  \bibitem{VSCode完全入門}
  リブロワークス\\
  Visual Studio Code 完全入門
  \bibitem{エラーをしながら}
  @hikozaru1202\\
  エラーしながら学ぶVSCodeにLatexを導入1,2,3\\
  \url{https://qiita.com/hikozaru1202/items/4189bfc52a99b7c32968}\\
  \url{https://qiita.com/hikozaru1202/items/befa7ddb6ea1b8920c92}\\
  \url{https://qiita.com/hikozaru1202/items/ce4c916f8d763b4006a9#latexmkrc%E3%81%AE%E4%BD%9C%E6%88%90}
  \bibitem{エラーの読み方}
  \verb|@Yarakashi_Kikohshi|\\
  エラーと警告を読むゾ\\
  \url{https://qiita.com/Yarakashi_Kikohshi/items/4ede21b06d094ad3b89e#-%E6%AD%A3%E3%81%97%E3%81%8F%E3%83%AD%E3%82%B0%E3%82%92-latex-workshop-%E3%81%AB%E5%BC%95%E3%81%8D%E6%B8%A1%E3%81%99}
  \bibitem{雑多な話題}
  \verb|@t_kemmochi|\\
  イマドキのLaTeXの書き方入門\\
  \url{https://qiita.com/t_kemmochi/items/78064daaa3903b7925ab#%E4%BB%98%E9%8C%B2-%E7%92%B0%E5%A2%83%E6%A7%8B%E7%AF%89}
  \bibitem{警告メッセージ}
  TeX Wiki\\
  LaTeX の警告メッセージ\\
  \url{https://texwiki.texjp.org/?LaTeX%20%E3%81%AE%E8%AD%A6%E5%91%8A%E3%83%A1%E3%83%83%E3%82%BB%E3%83%BC%E3%82%B8}
  \bibitem{エラーメッセージ}
  TeX Wiki\\
  LaTeX のエラーメッセージ\\
  \url{https://texwiki.texjp.org/?LaTeX%20%E3%81%AE%E3%82%A8%E3%83%A9%E3%83%BC%E3%83%A1%E3%83%83%E3%82%BB%E3%83%BC%E3%82%B8}
  \bibitem{CaTeX}
  san.com\\
  CaTeX(軽鳥/怪鳥)で快適 LaTeX ライフ in VSCode\\
  \url{https://konn-san.com/articles/2018-11-26-happy-latex-with-catex.html}
  \bibitem{ChkTeX}
  @skikkh\\
  VScodeで快適LaTeX環境を構築する方法\\
  \url{https://qiita.com/skikkh/items/707e8a5def368a69e9a6}
  \bibitem{LaTeX Utilities}
  TsubasaTakeda\\
  LaTeXをVScodeで書く(初心者のレポート向け)\\
  \url{https://qiita.com/TsubasaTakeda/items/bed2856c6bd427268144#%E6%8B%A1%E5%BC%B5%E6%A9%9F%E8%83%BD%E3%82%92%E8%BF%BD%E5%8A%A0}
  \bibitem{VScode 便利機能 マルチカーソル}
  @TomK\\
  VSCodeのマルチカーソル練習帳\\
  \url{https://qiita.com/TomK/items/3b1f5be07d708d7bd6c5}
  \bibitem{VScodeショートカットキー}
  12345\\
  VS Code の便利なショートカットキー\\
  \url{https://qiita.com/12345/items/64f4372fbca041e949d0}
  \bibitem{package}
  \verb|Yarakashi_Kikohshi|\\
  LaTeX でいろんなパッケージを usepackage する\\
  \url{https://qiita.com/Yarakashi_Kikohshi/items/97f9f920fb23974e0011}
  \bibitem{LaTeX workshop1}
  \verb|Yarakashi_Kikohshi|\\
  LaTeX Workshop を使いこなす\\
  \url{https://qiita.com/Yarakashi_Kikohshi/items/a9357dd469320ffb65a0}
  \bibitem{LaTeX Workshop2}
  \verb|Yarakashi_Kikohshi|\\
  LaTeX Workshop をもう少し使いこなす\\
  \url{https://qiita.com/Yarakashi_Kikohshi/items/1a275f2046b002e398b3}
  \bibitem{LaTeX Workshop3}
  \verb|Yarakashi_Kikohshi|\\
  LaTeX Workshop をもっと使いこなす\\
  \url{https://qiita.com/Yarakashi_Kikohshi/items/1f2225c7e28aad498998}
  \bibitem{LaTeX Workshop4}
  \verb|Yarakashi_Kikohshi|\\
  Zotero と LaTeX Workshop で bib ファイルを扱いこなす\\
  \url{https://qiita.com/Yarakashi_Kikohshi/items/8f720643543ba175f7cc}
  \bibitem{数式プレビュー}
  \verb|Yarakashi_Kikohshi|\\
  LaTeX Workshop の数式プレビューを使いこなす\\
  \url{https://qiita.com/Yarakashi_Kikohshi/items/4570bba51787e47a03c6}
  \bibitem{cloud LaTeX 連携}
  \verb|mtk_birdman's blog|\\
  【Windows】Visual Studio Code で Cloud LaTeX の実行環境を構築する(2022.02.16)\\
  \url{https://mtkbirdman.com/windows-visual-studio-code-cloud-latex-install}
  \bibitem{cloudlatex 利点と欠点}
  Yarakashi-Kikohshi\\
  VSCode で編集してCloud LaTeX でタイプセットする\\
  \url{https://gist.github.com/Yarakashi-Kikohshi/e554045b77d35bd132eb976034625023}

  
\end{thebibliography}

\end{document}
